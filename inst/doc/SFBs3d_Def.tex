\NeedsTeXFormat{LaTeX2e}
\documentclass[12pt,titlepage,oneside,a4paper]{article}
%\documentclass[12pt,titlepage,oneside,fleqn]{article}
\usepackage[ansi]{umlaute}
%\usepackage{german}
\usepackage{verbatim}
\usepackage{fancyvrb}
\usepackage{array}
\usepackage[active]{srcltx}
\usepackage[intlimits]{amsmath}
\usepackage{amsthm}
\usepackage{amssymb}
\usepackage[final]{graphicx}
\usepackage{float}
\usepackage{color}
\usepackage{Rd}%-> {url} and many more!
\usepackage{chicago}
\usepackage{hyperref}
  \definecolor{Blue}{rgb}{0,0,0.8}
  \definecolor{Red}{rgb}{0.7,0,0}
  \hypersetup{%
    backref,
    hyperindex,%
    colorlinks,%
    pagebackref,%
    linktocpage,%
    plainpages=false,%
    linkcolor=Blue,%
    citecolor=Blue,%
    urlcolor=Red,%
    pdfstartview=Fit,%
    pdfview={XYZ null null null}
    }


%% Definition der Seitengr��en, Abst�nde, etc. ================================
\setlength{\paperwidth}{21cm} % A4 Gr��e setzen
\setlength{\paperheight}{29.7cm}
\setlength{\oddsidemargin}{0.46cm} % Seitenrand links: ca. 3 cm
\setlength{\topmargin}{-0.5cm} % Seitenrand oben ca. 3 cm
\setlength{\headheight}{1cm}
\setlength{\headsep}{0cm}
\setlength{\footskip}{2cm} % Fu� vern�nftig
\setlength{\textwidth}{15cm} % Textbreite, so da� Seitenrand rechts ca. 3 cm
\setlength{\textheight}{22cm} % Texth�he vern�nftig

\setlength{\tabcolsep}{3mm}
\setlength{\doublerulesep}{0.2mm}
\parindent = 0em                     % Absatzeinr�ckung
\parskip = 2ex plus0.3ex minus0.3ex  % Absatzabstand
\renewcommand{\baselinestretch}{1.3} % Zeilenabstand
\sloppy % m�glichst wenig Trennen !
\raggedbottom % m�glichst sch�ne Seitenumbr�che

\setlength{\partopsep}{0mm} \setlength\topsep{0mm} \setlength\parsep{0mm}
%-------------------------------------------------------------

\renewcommand{\cite}[1]{\shortciteANP{#1}, \citeyearNP{#1}}
\renewcommand{\citeN}[1]{\shortciteN{#1}}

% Mathematikeinstellungen  ==================================================
\newcommand{\bmath}{\begin{eqnarray}}
\newcommand{\emath}{\end{eqnarray}}
\newcommand{\bmathn}{\begin{eqnarray*}}
\newcommand{\emathn}{\end{eqnarray*}}
\newcommand{\RR}{{\normalfont\textsf{R}}{}}%\newcommand{\RR}{{\bf R}}
\newcommand{\sdd}{\emph{scatterplot3d}}

\newcommand{\D}{\displaystyle}
\renewcommand{\epsilon}{\varepsilon}
\renewcommand{\R}{\mathbb{R}}
\newcommand{\C}{\mathbb{C}}
\newcommand{\N}{\mathbb{N}}
\newcommand{\Z}{\mathbb{Z}}
\newcommand{\Q}{\mathbb{Q}}

\newcommand{\bi}{\begin{itemize} \setlength\itemsep{0.5ex plus0.2ex minus0.3ex}}
\newcommand{\ei}{\end{itemize}}

